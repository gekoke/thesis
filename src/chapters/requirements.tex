\section{Funktsionaalsed nõuded}

Süsteemi funktsionaalsed nõuded tulenevad eeskätt kasutaja vajadustest. Süsteem peab olema võimeline toetama põhilisemaid kasutusvooge:

\begin{itemize}
  \item Rakendus peab lubama genereerida alamhulka riikliku õppekava matemaatika ülesannetest
  \item Rakendus peab lubama eksportida genereeritud ülesandeid formaadis, mis soosib paberkandjale printimist
  \item Rakendus peab lubama kasutajal määrata, mis ülesanded satuvad prinditud dokumendile
  \item Rakendus peab lubama kasutajal hallata juba genereeritud töid
  \item Rakendus peab lubama genereerida töödele vastuste lehti
\end{itemize}

\section{Mittefunktsionaalsed nõuded}

Süsteemi mittefunktsionaalseteks nõueteks on:

\begin{itemize}
  \item Rakendus peab olema kättesaadav eesti keeles
  \item Rakendus peab olema kättesaadav mistahes veebilehitsejast
  \item Rakenduse lähtekood peab olema täies mahus inglisekeelne (v.a tõlked)
  \item Rakenduse funktsionaalsus peab olema testitud
  \item Rakenduse töökindlus peab olema kindlaks tehtud katsekeskkonnas, mis on süsteemi tarbekeskkonnaga võimalikult sarnane
  \item Arenduse käigus loodud lähtekood uute funktsionaalsuste tarbeks peab läbima pidevat kvaliteedikontrolli ja integratsiooni ülejäänud lähtekoodiga, s.t järgima CI (\textit{Continuous Integration}) põhimõtteid (vt \ref{ci})
  \item Rakenduse tarbekeskkonda tarnimine peab toimuma pidevalt, s.t järgima CD (\textit{Continuous Deployment}) põhimõtteid (vt \ref{cd})
  \item Rakenduse arendus peab olema iga arendaja kohta võimalikult kiire ja produktiivne
  \item Rakenduse arenduse käigus tehtud otsused peavad olema võimalikult kiiresti vajaduse korral ümber lükatavad
  \item Rakenduse arendus peab olema minimaalselt rahaliselt kulukas
\end{itemize}

\subsection{Pideva integratsiooni põhimõtted}\label{ci}
Pidev integratsioon on põhimõtete ning võtete hulk, mis võimaldab:

\begin{itemize}
  \item automaatselt ja sagedaselt testida ja tootestada lähtekoodi \cite{gitlab-what-are-pipelines}
  \item sagedasi lähtekoodi mestimisi soosides vähendada mestekonflikte \cite{gitlab-what-is-ci}
  \item kiirendada arendusprotsessist saadud tagasiside saamist, ning tegeleda ilmnenud probleemidega kiiremalt \cite{gitlab-what-is-ci}
\end{itemize}

Leidub kaasaegseid tehnoloogiaid, mis on laialdases kasutuses \cite{github-features-actions} ning toetavad pideva integratsiooni põhimõtteid — vt nt. jaotist “\nameref{github-actions}”.

\subsection{Pideva tarnimise põhimõtted}\label{cd}
Pidev tarnimine on põhimõtete ning võtete hulk, mis võimaldab:

\begin{itemize}
  \item automaatselt ja sagedaselt tarnida rakendust tarbekeskkonda \cite{gitlab-what-is-cd}
  \item muuta rakenduse tarbekeskkonda tarnimist rutiinseks ja vähendada sellega kaasnevaid riske \cite{gitlab-cicd-fundamentals}
  \item taastada rakenduse töökindlus tarbekeskkonnas murede ilmnemisel kiiremini \cite{gitlab-benefits-of-cicd}
\end{itemize}

Leidub kaasaegseid tehnoloogiaid, mis on laialdases kasutuses \cite{github-features-actions} ning toetavad pideva tarnimise põhimõtteid — vt nt. jaotist “\nameref{github-actions}”.

