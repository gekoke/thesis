\section{Kasutaja tagasiside}

Töö jaoks küsiti tagasisidet gümnaasiumi matemaatikaõpetajalt, kellel paluti süsteemi kasutada ilma abita.

Kasutajal oli rakendust kasutades raskus leida üles ülesannete vastused. Peale töö loomist ja salvestamist ei suunatud kasutajat kuhugi edasi, mis võis tekitada kasutajas segadust. Töö genereerimise vaates oli küll kuvatud töö variandid, kuid puudus vastuseleht.

Kasutaja ei saanud aru rakenduses olevasest malli põhimõttest. Rakendus ei selgitanud kasutajale piisavalt hästi mis on mall ja kuidas see on seotud genereeritavate töödega.

Kasutaja soovis, et rakenduses oleks võimalik luua rohkem ülesandeid, mis oleksid sobilikud kümnenda klassi õpilastele. Rakenduses oli tagasiside saamise hetkel ainult kolm ülesandetüüpi, mis olid kümnenda klassi õpilaste tasemel liiga lihtsad.

\section{Täiustused}

Tagasiside põhjal tehti töös muudatusi, et parandada rakenduse kasutamise kogemust.

Selleks, et loodud töid ja nendega seotud vastuseid oleks lihtsam leida, loodi teatesõnum mida kuvatakse peale töö genereerimist. Teatesõnumile vajutades suunatakse kasutaja töö detailvaatesse, kus kasutaja saab eraldi avada erinevaid töö variante ning tööga seotud vastuselehte. Lisati ka töö genereerimise vaatesse võimalus kuvada vastuselehte, et vastused oleks kohe näha enne töö salvestamist.

Lisaks muudele täiustustele lisati rakendusele ka tagasiside jagamise vorm. Vorm on saadaval rakenduse päises oleva nupu all ning seda vajutades suunatakse kasutaja tagasiside jagamise vaatesse. Vaates on tekstikast, kuhu saab kasutaja soovitud tagasiside kirjutada ning nupp tagasiside saatmiseks.
