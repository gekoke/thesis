Lähiminevikus toimunud kaheksapäevaline õpetajate streik \cite{hm-opetajate-streik} on üha enam toonud vaatluse alla ületöötatuse, mis vaevab ühe olulisima ameti esindajaid.

Käesolev töö proovib eelmainitud puudusi leevendada, arendades välja rakenduse, mis võimaldab üldharidusõpetajatel veeta vähem aega ülesannete ja tööde koostamisega, jättes rohkem aega muudele tööalastele toimingutele.

\section{Töö eesmärgid}

Töö peamine eesmärk on arendada rakendus, mis loob matemaatikaülesandeid, säästmaks õpetajate aega. Matemaatikaülesanded peavad olema üldhariduskoolides õpetamise tarbeks kõlblikud.

Rakenduse algne skoop on piiratud ja hõlmab vaid keskkooli taseme matemaatika õppeaine jaoks ülesannete genereerimise funktsionaalsust. Eesmärk ei ole luua põhjalikku õpitaristut \textit{a la} Moodle.

Rakendus valmib tellimusprojekti käigus, mille tellijaks on Tallinna 21. Kooli keskkooli ja põhikooli matemaatikaõpetaja.

Töö edukalt valmimise tulemusena on valminud rakendus, mis:

\begin{itemize}
  \item on kasutajasõbralik
  \item võimaldab genereerida matemaatikaülesandeid
  \item võimaldab kasutajal hallata juba genereeritud ülesandeid, ning viia läbi enimlevinuid haldustoiminguid olemitega
  \item võimaldab ülesandeid alla tõmmata failivormingus, mis soosib paberkandjale printimist
\end{itemize}

