Töö tulemusena valmis rakendus Abiõpetaja, mis võimaldab kasutajal luua ning hallata matemaatika töid. Rakendus võimaldab koostada malle, millel on defineeritud ülesanded. Mallid on korduvkasutatavad ning käituvad kui raamistikud tööde genereerimiseks. Tööga koos genereeritakse ka vastuseleht.

Töö teostamisel saavutati kõik funktsionaalsed nõuded (vaata jaotis \ref{sec:functional-requirements}).

Kuigi töö käigus loodi võimalus genereerida alamhulka riikliku õppekava matemaatika ülesannetest, on see alamhulk ikkagi väike. Kuna tegelik ülesannete hulk on väga suur ei suutnud autorid luua märkimisväärset lahendust probleemile. Sellegipoolest on võimalik rakendust kasutada, et luua lihtsamaid ülesandeid.

Töö teostamisel lähtuti ka mittefunktsionaalsetest nõuetest (vaata jaotis \ref{sec:nonfunctional-requirements}), mis said kõik täielikult või osaliselt täidetud. Osaliselt täidetud nõue oli rakenduse funktsionaalsuse testimine automaattestidega. Kuna rakenduse funktsionaalsuste ja kõikvõimalike kasutusviiside koguse tõttu ei ole võimalik või praktiline kõike testida, ei kirjutatud iga olukorra jaoks teste. Sellegipoolest on rakenduses tagatud põhifunktsioonide automaattestide kattumus.
