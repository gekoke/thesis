Käesolev peatükk kirjeldab tehtud töö puudusi ja võimalike samme, et edendada käsitletud teemade edasist uurimist.

\section{Vabavarasse panustamine}

Jaotises \ref{subsec:hunspell} kirjeldati Hunspelli eestikeelse sõnastiku pakendamist. Paraku oli sõnastik aastast 2003 ning saadaval vaid \texttt{latin-1} (ISO/IEC 8859-1 kirjeldatud) ja \texttt{latin-9} (ISO/IEC 8859-15 kirjeldatud) kodeeringus. See tähendab, et sõnastiku ei tõlgendata korrektselt kui seda rakendada sisendteksti peale, mis on tänapäeval levinud \texttt{UTF-8} kodeeringus ning sisaldab tähti \texttt{š} või \texttt{ž}.

Olukorra parandamiseks tuleks sõnastik ümber kodeerida tänapäeval üldtuntud \texttt{UTF-8} kodeeringusse, ning alles seejärel pakendada.


