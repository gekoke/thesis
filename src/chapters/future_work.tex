Käesolev peatükk kirjeldab tehtud töö puudusi ja võimalike samme, et edendada käsitletud teemade edasist uurimist.

\section{Ülesannete hulga laiendamine}

Töö raames sai implementeeritud genereerimine vaid järgnevate ülesandeklasside jaoks:

\begin{itemize}
  \item lineaarvõrratused
  \item ruutvõrratused
  \item murdvõrratused
\end{itemize}

Tulemus vastab jaotises \ref{sec:functional-requirements} kirjeldatud nõutele. Sellegipoolest on otstarbekas tulevikus implementeerida veel riiklikus õppekavas olevaid ülesandeklasse. Selline toimimine võimaldab hinnata erinevate ülesandeklasside genereerimise kasutajasõbralikku liidesesse pakendamise võimalikust ning keerukust.

\section{Kasutajaskonna laiendamine}

Tehtud töö eesmärk oli selge ja suunatud — hõlbustada õpetajate tööd. Ometi võiks töös arendatud leida laialdasemat rakendust.

Ülesannete ja vastuste genereerimine ei peaks ilmtingimata olema suunatud vaid õpetajatele. Rakenduse võimekust võiks saada rakendada ka näiteks õpilased harjutamise ning kordamise tarbeks. Niiviisi saaks suurim hulk inimesi kasu käesoleva töö raames arendatust.
\section{Digitaliseerimine}

Aastal 2025 muutuvad Eesti põhikoolide lõpueksamid digitaalseks \cite{digital-exams}. Võib ette näha, et aja möödudes tabab sarnane saatus igasuguseid muid õpiprotsesse.

Tõsiasjad seavad kahtluse alla töö fookuse paberkandjale kui ülesannete ja vastuste vahendajale. Edasine töö võiks uurida rakenduse kohandamist, et toetada ülesannetega täies mahus digitaalselt ümberkäimist. Selline talitlemine võiks ära kasutada juba arendatud süsteemi tuuma, ent pakkuda seda teistsuguses kontekstis.

\section{Vabavarasse panustamine}

Jaotises \ref{subsec:hunspell} kirjeldati Hunspelli eestikeelse sõnastiku pakendamist. Paraku oli sõnastik aastast 2003 ning saadaval vaid \texttt{latin-1} (ISO/IEC 8859-1 kirjeldatud) ja \texttt{latin-9} (ISO/IEC 8859-15 kirjeldatud) kodeeringus. See tähendab, et sõnastiku ei tõlgendata korrektselt kui seda rakendada sisendteksti peale, mis on tänapäeval levinud \texttt{UTF-8} kodeeringus ning sisaldab tähti \texttt{š} või \texttt{ž}.

Olukorra parandamiseks tuleks sõnastik ümber kodeerida tänapäeval üldtuntud \texttt{UTF-8} kodeeringusse, ning alles seejärel pakendada.

