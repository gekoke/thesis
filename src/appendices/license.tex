\newcommand{\licenseFootnote}{Lihtlitsents ei kehti juurdepääsupiirangu kehtivuse ajal vastavalt üliõpilase taotlusele lõputööle juurdepää- supiirangu kehtestamiseks, mis on allkirjastatud teaduskonna dekaani poolt, välja arvatud ülikooli õigus lõputööd reprodutseerida üksnes säilitamise eesmärgil. Kui lõputöö on loonud kaks või enam isikut oma ühise loomingulise tegevusega ning lõputöö kaas- või ühisautor(id) ei ole andnud lõputööd kaitsvale üliõpilasele kindlaksmääratud tähtajaks nõusolekut lõputöö reprodutseerimiseks ja avalikustamiseks vastavalt lihtlitsentsi punktidele 1.1. ja 1.2, siis lihtlitsents nimetatud tähtaja jooksul ei kehti.}

\addcontentsline{toc}{chapter}{Lisa 1 – Lihtlitsents lõputöö reprodutseerimiseks ja lõputöö üldsusele kättesaadavaks tegemiseks}\label{chapter:license}
{\let\clearpage\relax\chapter*{Lisa 1 – Lihtlitsents lõputöö reprodutseerimiseks ja lõputöö üldsusele kättesaadavaks tegemiseks\footnote{\licenseFootnote}}}

Meie, \authorNames{}

\begin{enumerate}[label*=\arabic*.]
    \item Anname Tallinna Tehnikaülikoolile tasuta loa (lihtlitsentsi) enda loodud teose “\thesisTitle{}”, mille juhendaja on \supervisorName{}.
    \begin{enumerate}[label*=\arabic*.]
        \item reprodutseerimiseks lõputöö säilitamise ja elektroonse avaldamise eesmärgil, sh Tallinna Tehnikaülikooli raamatukogu digikogusse lisamise eesmärgil kuni autoriõiguse kehtivuse tähtaja lõppemiseni;
        \item üldsusele kättesaadavaks tegemiseks Tallinna Tehnikaülikooli veebikeskkonna kaudu, sealhulgas Tallinna Tehnikaülikooli raamatukogu digikogu kaudu kuni autoriõiguse kehtivuse tähtaja lõppemiseni.
    \end{enumerate}
    \item Oleme teadlikud, et käesoleva lihtlitsentsi punktis 1 nimetatud õigused jäävad alles ka autoritele.
    \item Kinnitame, et lihtlitsentsi andmisega ei rikuta teiste isikute intellektuaalomandi ega isikuandmete kaitse seadusest ning muudest õigusaktidest tulenevaid õigusi.
\end{enumerate}

01.05.2024

