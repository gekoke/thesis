\begin{longtable}{p{4cm}p{10cm}}
AMI&\textit{Amazon Machine Image} — Amazoni pilvteenuste jaoks kasutatud virtuaalmasinatõmmis\\
CD&\textit{Continuous Deployment} või \textit{Continuous Delivery} — pidev tarnimine\\
CI&\textit{Continuous Integration} — pidev integratsioon\\
\textit{Forge}&veebiliidesega koostööplatvorm, mis tugineb versioonihaldustarkvarale\\
\textit{Fork}&tarkvaraprojekti kloon, millel alustatakse eraldiseisvat või hargnevat arendustööd\\
\textit{Commit}&Lähtekoodi salvestatud olek versioonihalduses\\
Linuxi distributsioon&Operatsioonisüsteem, mis kasutab Linuxi tuuma, kuid võib muu kaasatud tarkvara poolest erineda\\
MVC&\textit{Model-View-Controller} — tarkvara arhitektuurimuster, eriti veebirakenduste kohta\\
Nix&paketihaldur, mis taotleb korratavust ning tugineb puhastele funktsioonidele\\
\textit{Nix Language}& Nixi programmeerimiskeel — programmeerimiskeel, mida kasutatakse Nixi tarkvarapakettide defineerimiseks\\
Nixpkgs&Nixi paketihalduri peamine paketiarhiiv ja Nixi programmeerimiskeele \textit{de facto} standardteek\\
ORM&\emph{Object Relational Mapper} — objekt-relatsiooniline kaardistaja\\
PEP&\textit{Python Enhancement Proposal} — ametlik ettepanek, mille abil tuuakse olulisi muudatusi Pythoni programmeerimiskeelde või selle taristusse\\
Puhas funktsioon&Funktsioon, mis ei tekita kõrvalmõjusid — erinevalt üldlevinud funktsiooni mõistest programmeerimises\\
Python&Laialdaselt kasutatud üldotstarbe programmeerimiskeel\\
\end{longtable}
\addtocounter{table}{-1} 

