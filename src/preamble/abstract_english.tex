The aim of this thesis was to create the web application Abiõpetaja for generating maths tests and to explore the possibilities of using the Python library SymPy to generate maths problems. The application is meant to decrease the work load on teachers and to simplify the creation of tests.

The main issues during the development of the application were the generation of user—friendly PDF files of tests and the generation of random problems.

The result of this thesis is the successful creation of the web application which allowed the generation of a subset of maths problems from the national curriculum.

The thesis is written in Estonian and is \total{page} pages long, including 9 chapters, \total{figure} figures and \total{table} tables.

